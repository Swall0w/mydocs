% Generated by Sphinx.
\def\sphinxdocclass{report}
\newif\ifsphinxKeepOldNames \sphinxKeepOldNamestrue
\documentclass[letterpaper,10pt,dvipdfmx]{sphinxmanual}
\usepackage{iftex}

\ifPDFTeX
  \usepackage[utf8]{inputenc}
\fi
\ifdefined\DeclareUnicodeCharacter
  \DeclareUnicodeCharacter{00A0}{\nobreakspace}
\fi
\usepackage{cmap}
\usepackage[T1]{fontenc}
\usepackage{amsmath,amssymb,amstext}

\usepackage{times}

\usepackage{longtable}
\usepackage{sphinx}
\usepackage{multirow}
\usepackage{eqparbox}


\renewcommand{\figurename}{図 }
\renewcommand{\tablename}{TABLE }
\SetupFloatingEnvironment{literal-block}{name=LIST }

\def\pageautorefname{ページ}

\setcounter{tocdepth}{1}


\title{MyOutput Documentation}
\date{11月 08, 2016}
\release{0.0.1}
\author{Masato}
\newcommand{\sphinxlogo}{}
\renewcommand{\releasename}{リリース}
\makeindex

\makeatletter
\def\PYG@reset{\let\PYG@it=\relax \let\PYG@bf=\relax%
    \let\PYG@ul=\relax \let\PYG@tc=\relax%
    \let\PYG@bc=\relax \let\PYG@ff=\relax}
\def\PYG@tok#1{\csname PYG@tok@#1\endcsname}
\def\PYG@toks#1+{\ifx\relax#1\empty\else%
    \PYG@tok{#1}\expandafter\PYG@toks\fi}
\def\PYG@do#1{\PYG@bc{\PYG@tc{\PYG@ul{%
    \PYG@it{\PYG@bf{\PYG@ff{#1}}}}}}}
\def\PYG#1#2{\PYG@reset\PYG@toks#1+\relax+\PYG@do{#2}}

\expandafter\def\csname PYG@tok@ni\endcsname{\let\PYG@bf=\textbf\def\PYG@tc##1{\textcolor[rgb]{0.84,0.33,0.22}{##1}}}
\expandafter\def\csname PYG@tok@sh\endcsname{\def\PYG@tc##1{\textcolor[rgb]{0.25,0.44,0.63}{##1}}}
\expandafter\def\csname PYG@tok@sd\endcsname{\let\PYG@it=\textit\def\PYG@tc##1{\textcolor[rgb]{0.25,0.44,0.63}{##1}}}
\expandafter\def\csname PYG@tok@kc\endcsname{\let\PYG@bf=\textbf\def\PYG@tc##1{\textcolor[rgb]{0.00,0.44,0.13}{##1}}}
\expandafter\def\csname PYG@tok@na\endcsname{\def\PYG@tc##1{\textcolor[rgb]{0.25,0.44,0.63}{##1}}}
\expandafter\def\csname PYG@tok@vc\endcsname{\def\PYG@tc##1{\textcolor[rgb]{0.73,0.38,0.84}{##1}}}
\expandafter\def\csname PYG@tok@kr\endcsname{\let\PYG@bf=\textbf\def\PYG@tc##1{\textcolor[rgb]{0.00,0.44,0.13}{##1}}}
\expandafter\def\csname PYG@tok@c1\endcsname{\let\PYG@it=\textit\def\PYG@tc##1{\textcolor[rgb]{0.25,0.50,0.56}{##1}}}
\expandafter\def\csname PYG@tok@kn\endcsname{\let\PYG@bf=\textbf\def\PYG@tc##1{\textcolor[rgb]{0.00,0.44,0.13}{##1}}}
\expandafter\def\csname PYG@tok@il\endcsname{\def\PYG@tc##1{\textcolor[rgb]{0.13,0.50,0.31}{##1}}}
\expandafter\def\csname PYG@tok@s\endcsname{\def\PYG@tc##1{\textcolor[rgb]{0.25,0.44,0.63}{##1}}}
\expandafter\def\csname PYG@tok@m\endcsname{\def\PYG@tc##1{\textcolor[rgb]{0.13,0.50,0.31}{##1}}}
\expandafter\def\csname PYG@tok@cp\endcsname{\def\PYG@tc##1{\textcolor[rgb]{0.00,0.44,0.13}{##1}}}
\expandafter\def\csname PYG@tok@nb\endcsname{\def\PYG@tc##1{\textcolor[rgb]{0.00,0.44,0.13}{##1}}}
\expandafter\def\csname PYG@tok@bp\endcsname{\def\PYG@tc##1{\textcolor[rgb]{0.00,0.44,0.13}{##1}}}
\expandafter\def\csname PYG@tok@gh\endcsname{\let\PYG@bf=\textbf\def\PYG@tc##1{\textcolor[rgb]{0.00,0.00,0.50}{##1}}}
\expandafter\def\csname PYG@tok@nl\endcsname{\let\PYG@bf=\textbf\def\PYG@tc##1{\textcolor[rgb]{0.00,0.13,0.44}{##1}}}
\expandafter\def\csname PYG@tok@sc\endcsname{\def\PYG@tc##1{\textcolor[rgb]{0.25,0.44,0.63}{##1}}}
\expandafter\def\csname PYG@tok@gi\endcsname{\def\PYG@tc##1{\textcolor[rgb]{0.00,0.63,0.00}{##1}}}
\expandafter\def\csname PYG@tok@ss\endcsname{\def\PYG@tc##1{\textcolor[rgb]{0.32,0.47,0.09}{##1}}}
\expandafter\def\csname PYG@tok@ow\endcsname{\let\PYG@bf=\textbf\def\PYG@tc##1{\textcolor[rgb]{0.00,0.44,0.13}{##1}}}
\expandafter\def\csname PYG@tok@s2\endcsname{\def\PYG@tc##1{\textcolor[rgb]{0.25,0.44,0.63}{##1}}}
\expandafter\def\csname PYG@tok@mb\endcsname{\def\PYG@tc##1{\textcolor[rgb]{0.13,0.50,0.31}{##1}}}
\expandafter\def\csname PYG@tok@cs\endcsname{\def\PYG@tc##1{\textcolor[rgb]{0.25,0.50,0.56}{##1}}\def\PYG@bc##1{\setlength{\fboxsep}{0pt}\colorbox[rgb]{1.00,0.94,0.94}{\strut ##1}}}
\expandafter\def\csname PYG@tok@gs\endcsname{\let\PYG@bf=\textbf}
\expandafter\def\csname PYG@tok@gt\endcsname{\def\PYG@tc##1{\textcolor[rgb]{0.00,0.27,0.87}{##1}}}
\expandafter\def\csname PYG@tok@mf\endcsname{\def\PYG@tc##1{\textcolor[rgb]{0.13,0.50,0.31}{##1}}}
\expandafter\def\csname PYG@tok@ch\endcsname{\let\PYG@it=\textit\def\PYG@tc##1{\textcolor[rgb]{0.25,0.50,0.56}{##1}}}
\expandafter\def\csname PYG@tok@nn\endcsname{\let\PYG@bf=\textbf\def\PYG@tc##1{\textcolor[rgb]{0.05,0.52,0.71}{##1}}}
\expandafter\def\csname PYG@tok@go\endcsname{\def\PYG@tc##1{\textcolor[rgb]{0.20,0.20,0.20}{##1}}}
\expandafter\def\csname PYG@tok@cm\endcsname{\let\PYG@it=\textit\def\PYG@tc##1{\textcolor[rgb]{0.25,0.50,0.56}{##1}}}
\expandafter\def\csname PYG@tok@gp\endcsname{\let\PYG@bf=\textbf\def\PYG@tc##1{\textcolor[rgb]{0.78,0.36,0.04}{##1}}}
\expandafter\def\csname PYG@tok@gu\endcsname{\let\PYG@bf=\textbf\def\PYG@tc##1{\textcolor[rgb]{0.50,0.00,0.50}{##1}}}
\expandafter\def\csname PYG@tok@sx\endcsname{\def\PYG@tc##1{\textcolor[rgb]{0.78,0.36,0.04}{##1}}}
\expandafter\def\csname PYG@tok@kd\endcsname{\let\PYG@bf=\textbf\def\PYG@tc##1{\textcolor[rgb]{0.00,0.44,0.13}{##1}}}
\expandafter\def\csname PYG@tok@vi\endcsname{\def\PYG@tc##1{\textcolor[rgb]{0.73,0.38,0.84}{##1}}}
\expandafter\def\csname PYG@tok@ne\endcsname{\def\PYG@tc##1{\textcolor[rgb]{0.00,0.44,0.13}{##1}}}
\expandafter\def\csname PYG@tok@s1\endcsname{\def\PYG@tc##1{\textcolor[rgb]{0.25,0.44,0.63}{##1}}}
\expandafter\def\csname PYG@tok@ge\endcsname{\let\PYG@it=\textit}
\expandafter\def\csname PYG@tok@nv\endcsname{\def\PYG@tc##1{\textcolor[rgb]{0.73,0.38,0.84}{##1}}}
\expandafter\def\csname PYG@tok@nc\endcsname{\let\PYG@bf=\textbf\def\PYG@tc##1{\textcolor[rgb]{0.05,0.52,0.71}{##1}}}
\expandafter\def\csname PYG@tok@sr\endcsname{\def\PYG@tc##1{\textcolor[rgb]{0.14,0.33,0.53}{##1}}}
\expandafter\def\csname PYG@tok@vg\endcsname{\def\PYG@tc##1{\textcolor[rgb]{0.73,0.38,0.84}{##1}}}
\expandafter\def\csname PYG@tok@nd\endcsname{\let\PYG@bf=\textbf\def\PYG@tc##1{\textcolor[rgb]{0.33,0.33,0.33}{##1}}}
\expandafter\def\csname PYG@tok@no\endcsname{\def\PYG@tc##1{\textcolor[rgb]{0.38,0.68,0.84}{##1}}}
\expandafter\def\csname PYG@tok@mh\endcsname{\def\PYG@tc##1{\textcolor[rgb]{0.13,0.50,0.31}{##1}}}
\expandafter\def\csname PYG@tok@se\endcsname{\let\PYG@bf=\textbf\def\PYG@tc##1{\textcolor[rgb]{0.25,0.44,0.63}{##1}}}
\expandafter\def\csname PYG@tok@err\endcsname{\def\PYG@bc##1{\setlength{\fboxsep}{0pt}\fcolorbox[rgb]{1.00,0.00,0.00}{1,1,1}{\strut ##1}}}
\expandafter\def\csname PYG@tok@mo\endcsname{\def\PYG@tc##1{\textcolor[rgb]{0.13,0.50,0.31}{##1}}}
\expandafter\def\csname PYG@tok@nt\endcsname{\let\PYG@bf=\textbf\def\PYG@tc##1{\textcolor[rgb]{0.02,0.16,0.45}{##1}}}
\expandafter\def\csname PYG@tok@cpf\endcsname{\let\PYG@it=\textit\def\PYG@tc##1{\textcolor[rgb]{0.25,0.50,0.56}{##1}}}
\expandafter\def\csname PYG@tok@gr\endcsname{\def\PYG@tc##1{\textcolor[rgb]{1.00,0.00,0.00}{##1}}}
\expandafter\def\csname PYG@tok@mi\endcsname{\def\PYG@tc##1{\textcolor[rgb]{0.13,0.50,0.31}{##1}}}
\expandafter\def\csname PYG@tok@o\endcsname{\def\PYG@tc##1{\textcolor[rgb]{0.40,0.40,0.40}{##1}}}
\expandafter\def\csname PYG@tok@sb\endcsname{\def\PYG@tc##1{\textcolor[rgb]{0.25,0.44,0.63}{##1}}}
\expandafter\def\csname PYG@tok@w\endcsname{\def\PYG@tc##1{\textcolor[rgb]{0.73,0.73,0.73}{##1}}}
\expandafter\def\csname PYG@tok@k\endcsname{\let\PYG@bf=\textbf\def\PYG@tc##1{\textcolor[rgb]{0.00,0.44,0.13}{##1}}}
\expandafter\def\csname PYG@tok@c\endcsname{\let\PYG@it=\textit\def\PYG@tc##1{\textcolor[rgb]{0.25,0.50,0.56}{##1}}}
\expandafter\def\csname PYG@tok@si\endcsname{\let\PYG@it=\textit\def\PYG@tc##1{\textcolor[rgb]{0.44,0.63,0.82}{##1}}}
\expandafter\def\csname PYG@tok@gd\endcsname{\def\PYG@tc##1{\textcolor[rgb]{0.63,0.00,0.00}{##1}}}
\expandafter\def\csname PYG@tok@nf\endcsname{\def\PYG@tc##1{\textcolor[rgb]{0.02,0.16,0.49}{##1}}}
\expandafter\def\csname PYG@tok@kt\endcsname{\def\PYG@tc##1{\textcolor[rgb]{0.56,0.13,0.00}{##1}}}
\expandafter\def\csname PYG@tok@kp\endcsname{\def\PYG@tc##1{\textcolor[rgb]{0.00,0.44,0.13}{##1}}}

\def\PYGZbs{\char`\\}
\def\PYGZus{\char`\_}
\def\PYGZob{\char`\{}
\def\PYGZcb{\char`\}}
\def\PYGZca{\char`\^}
\def\PYGZam{\char`\&}
\def\PYGZlt{\char`\<}
\def\PYGZgt{\char`\>}
\def\PYGZsh{\char`\#}
\def\PYGZpc{\char`\%}
\def\PYGZdl{\char`\$}
\def\PYGZhy{\char`\-}
\def\PYGZsq{\char`\'}
\def\PYGZdq{\char`\"}
\def\PYGZti{\char`\~}
% for compatibility with earlier versions
\def\PYGZat{@}
\def\PYGZlb{[}
\def\PYGZrb{]}
\makeatother

\renewcommand\PYGZsq{\textquotesingle}

\begin{document}

\maketitle
\tableofcontents
\phantomsection\label{index::doc}


Contents:


\chapter{機械学習のまとめ}
\label{learned_machine_learning:id1}\label{learned_machine_learning::doc}\label{learned_machine_learning:welcome-to-myoutput-s-documentation}\begin{quote}\begin{description}
\item[{著者}] \leavevmode
Masato

\end{description}\end{quote}


\section{概要}
\label{learned_machine_learning:id2}
機械学習に必要な知識についてまとめてあります。参考文献葉は
\#. Christopher M. Bishop ``Pattern Recognition and Machine Learning''
\#. 機械学習プロフェッショナルシリーズ
からまとめています。現在は深層学習に向けた知識をまとめるので、主に取り扱っているのはニューラルネットワークです。


\section{ニューラルネットワーク}
\label{learned_machine_learning:id3}\begin{align*}\!\begin{aligned}
(a + b)^2 = a^2 + 2ab + b^2\\
(a - b)^2 = a^2 - 2ab + b^2\\
\end{aligned}\end{align*}
こんな感じに書いていきます。
逆誤差伝版については、今後書いていきます。


\chapter{数学}
\label{learned_math:id1}\label{learned_math::doc}\begin{quote}\begin{description}
\item[{著者}] \leavevmode
Masato

\end{description}\end{quote}


\section{概要}
\label{learned_math:id2}
数学について学んだことをまとめてます。取り扱う分野は主に線形代数と確率統計になります。


\section{確率統計}
\label{learned_math:id3}\begin{align*}\!\begin{aligned}
(a + b)^2 = a^2 + 2ab + b^2\\
(a - b)^2 = a^2 - 2ab + b^2\\
\end{aligned}\end{align*}
こんな感じに書いていきます。


\section{線形代数}
\label{learned_math:id4}

\chapter{アルゴリズム}
\label{learned_algorithm:id1}\label{learned_algorithm::doc}\begin{quote}\begin{description}
\item[{著者}] \leavevmode
Masato

\end{description}\end{quote}


\section{概要}
\label{learned_algorithm:id2}
アルゴリズムはについてまとめてます。数値計算などのカテゴリー分けしているため、それぞれ必要なものを見てください。。


\chapter{Python}
\label{learned_programing_python:python}\label{learned_programing_python::doc}\begin{quote}\begin{description}
\item[{著者}] \leavevmode
Masato

\end{description}\end{quote}


\section{概要}
\label{learned_programing_python:id1}
Python言語で役に立ったこと、つまったことについてまとめます。ライブラリの簡単な使い方などについてもまとめました。
ディープラーニングで使うNumpyや文字列処理についてもまとめてあるので、確認しながら仕様してください。明記していない限りはPython3での使用でまとめています。


\section{Anaconda}
\label{learned_programing_python:anaconda}

\section{Sphinx}
\label{learned_programing_python:sphinx}

\section{Numpy}
\label{learned_programing_python:numpy}

\section{リスト}
\label{learned_programing_python:id2}

\chapter{C++}
\label{learned_programing_cpp::doc}\label{learned_programing_cpp:c}\begin{quote}\begin{description}
\item[{著者}] \leavevmode
Masato

\end{description}\end{quote}


\section{概要}
\label{learned_programing_cpp:id1}
C++でのプログラミングについてまとめます。アルゴリズムは別にまとめてるため、こちらは言語仕様についてが中心になります。


\chapter{Java Script}
\label{learned_programing_js:java-script}\label{learned_programing_js::doc}\begin{quote}\begin{description}
\item[{著者}] \leavevmode
Masato

\end{description}\end{quote}


\section{概要}
\label{learned_programing_js:id1}
Java Scriptについてまとめます。文法の他、node.jsやd3.jsなどのライブラリーについてもまとめていきます。


\section{文法}
\label{learned_programing_js:id2}

\section{node.js}
\label{learned_programing_js:node-js}

\section{D3.js}
\label{learned_programing_js:d3-js}

\chapter{Arch Linux}
\label{learned_linux_archlinux:arch-linux}\label{learned_linux_archlinux::doc}\begin{quote}\begin{description}
\item[{著者}] \leavevmode
Masato

\end{description}\end{quote}


\section{概要}
\label{learned_linux_archlinux:id1}
Linuxのディストリビューションの一つであるArch Linuxについてインストールに必要なものから、インストール方法、
実際に使っているアプリケーションにいたるまで、やってきたことをまとめます。


\section{インストール}
\label{learned_linux_archlinux:id2}

\section{起動しないときに}
\label{learned_linux_archlinux:id3}
カーネルアップデート後に起動しなくなったときに、以下の手順でカーネルの再構築を試して見てください。
起動しなくなったLinuxが存在するrootパーティションをマウントする。例えば、/mnt/archにマウントする。:

\begin{Verbatim}[commandchars=\\\{\}]
\PYGZdl{} mkdir /mnt/arch
\PYGZdl{} mount /dev/sda1 /mnt/arch
\end{Verbatim}

/bootや/varが別のパーティションがある場合は、マウントする。例えば、:

\begin{Verbatim}[commandchars=\\\{\}]
\PYGZdl{} mount /dev/sda2 /mnt/arch/boot
\end{Verbatim}

など。
さらに、APIファイルシステムをマウントする。:

\begin{Verbatim}[commandchars=\\\{\}]
\PYGZdl{} cd /mnt/arch
\PYGZdl{} mount \PYGZhy{}t proc proc proc/
\PYGZdl{} mount \PYGZhy{}\PYGZhy{}rbind /sys sys/
\PYGZdl{} mount \PYGZhy{}\PYGZhy{}rbind /dev dev/
\end{Verbatim}

ファイルシステムのマウントができたら、chrootする。:

\begin{Verbatim}[commandchars=\\\{\}]
\PYGZdl{} chroot /mnt/arch /bin/bash
\end{Verbatim}

必要であれば、ネットの情報をコピーする。※追記

chrootができたら、カーネルイメージを作り直す。イメージを作る前に、udevとmkinitcpioを再インストールする。:

\begin{Verbatim}[commandchars=\\\{\}]
\PYGZdl{} pacman \PYGZhy{}Syy
\PYGZdl{} pacman \PYGZhy{}Syu
\PYGZdl{} pacman \PYGZhy{}S udev
\PYGZdl{} pacman \PYGZhy{}S mkinitcpio
\PYGZdl{} pacman \PYGZhy{}S linux
\PYGZdl{} mkinitcpio \PYGZhy{}p linux
\end{Verbatim}

カーネルイメージの再構成が成功したら、chrootを抜けだし、再起動する。:

\begin{Verbatim}[commandchars=\\\{\}]
\PYGZdl{} exit
\PYGZdl{} cd /
\PYGZdl{} umount \PYGZhy{}\PYGZhy{}recursive /mnt/arch/
\PYGZdl{} reboot
\end{Verbatim}

これで大抵OSが起動するようになる。Arch Linuxが起動しなくなるのは、アップデート時にカーネルイメージがうまく再構成できなかったり、
udevやmkinitcpioが上手くアップデートできないときに起こる。


\chapter{Linux サーバー関連}
\label{learned_linux_server::doc}\label{learned_linux_server:linux}\begin{quote}\begin{description}
\item[{著者}] \leavevmode
Masato

\end{description}\end{quote}


\section{概要}
\label{learned_linux_server:id1}
データベースの構築から、インストール、ipの固定までのhow to をまとめていきます。


\section{データベース}
\label{learned_linux_server:id2}

\section{Linux バージョン確認}
\label{learned_linux_server:id3}

\subsection{Cent OS}
\label{learned_linux_server:cent-os}
Cent OSのバージョン確認:

\begin{Verbatim}[commandchars=\\\{\}]
\PYGZdl{} cat /etc/redhat\PYGZhy{}releace
CentOS release 6.8 (Final)
\end{Verbatim}


\subsection{アーキテクチャを確認する。}
\label{learned_linux_server:id4}
OSのバージョン確認コマンド:

\begin{Verbatim}[commandchars=\\\{\}]
\PYGZdl{} arch
X86\PYGZus{}64 \PYGZlt{}\PYGZhy{} 64bitの場合
i686   \PYGZlt{}\PYGZhy{} 32bitの場合
\PYGZdl{} uname \PYGZhy{}a
\end{Verbatim}


\section{FTPコマンドでファイル転送}
\label{learned_linux_server:ftp}
FTPコマンドは、ホスト名とポート番号を指定して起動します。すると、ユーザ名とパスワードを聞いてくるので入力します。ユーザ認証がOKな場合は各種FTPコマンドが使えます。

\noindent\begin{tabulary}{\linewidth}{|L|L|}
\hline
\textsf{\relax 
fptコマンド
\unskip}\relax &\textsf{\relax 
機能
\unskip}\relax \\
\hline
ftp
&
FTPを起動する。
\\
\hline
ls
&
リモートのファイルの一覧を表示
\\
\hline
pwd
&
リモートのカレントディレクトリを表示
\\
\hline
cd
&
リモートのカレント作業ディレクトリを表示
\\
\hline
mkdir
&
リモートのディレクトリを作成
\\
\hline
!ls
&
ローカルのファイルの一覧を表示
\\
\hline
!pwd
&
ローカルのカレントディレクトリを表示
\\
\hline
!cd
&
ローカルのカレント作業ディレクトリを移動
\\
\hline
get
&
ファイルをダウンロード
\\
\hline
mget
&
複数のファイルをダウンロード
\\
\hline
put
&
ファイルをアップロード
\\
\hline
mput
&
複数のファイルをアップロード
\\
\hline
passive
&
Passiveモードの切り替え
\\
\hline
bin /binary
&
バイナリ転送モードに切り替え
\\
\hline
asc /ascii
&
ASCII転送モードに切り替え
\\
\hline
bye /exit /quit
&
FTPコマンド終了
\\
\hline\end{tabulary}



\section{Ubuntuにpukiwikiインストール}
\label{learned_linux_server:ubuntupukiwiki}

\subsection{Apach2.4 \& PHP インストール}
\label{learned_linux_server:apach2-4-php}
普通にapt-getする。:

\begin{Verbatim}[commandchars=\\\{\}]
\PYGZdl{} sudo apt\PYGZhy{}get install apache2
\PYGZdl{} sudo apt\PYGZhy{}get install php
\end{Verbatim}

/var/www/htmlがデフォルトのパブリックフォルダとなっている。apt-getでインストール後は自動的にサーバプロセスが起動する。ブラウザで、:

\begin{Verbatim}[commandchars=\\\{\}]
\PYG{n}{http}\PYG{p}{:}\PYG{o}{/}\PYG{o}{/}\PYG{p}{[}\PYG{n}{サーバー名}\PYG{p}{]}\PYG{o}{/}
\PYG{n}{http}\PYG{p}{:}\PYG{o}{/}\PYG{o}{/}\PYG{n}{localhost}\PYG{o}{/}\PYG{n}{index}\PYG{o}{.}\PYG{n}{html}
\end{Verbatim}

へアクセスし、Apacheのデフォルトページが表示されればOK


\subsection{PukiWiki1.5.0を設置}
\label{learned_linux_server:pukiwiki1-5-0}
utf8版のPukiWikiを使う。WikiのURLは、\url{http:/}/{[}サーバー名{]}/wiki とする。zipファイルなので、unzipする。:

\begin{Verbatim}[commandchars=\\\{\}]
\PYGZdl{} sudo apt\PYGZhy{}get install unzip
\PYGZdl{} cd /var/www/html/
\PYGZdl{} sudo unzip pukiwiki\PYGZhy{}1\PYGZus{}5\PYGZus{}0\PYGZus{}utf8.zip
\PYGZdl{} sudo mv pukiwiki\PYGZhy{}1\PYGZus{}5\PYGZus{}0\PYGZus{}utf8 wiki
\end{Verbatim}

Wikiの実効権限をサーバプロセスと同じ、www-dataにする。:

\begin{Verbatim}[commandchars=\\\{\}]
\PYGZdl{} chwon \PYGZhy{}R www\PYGZhy{}data.www\PYGZhy{}data /var/www/html/wiki
\end{Verbatim}

これで \url{http:/}/{[}サーバ名{]}/wiki にアクセスし、Pukiwikiのデフォルトページが表示される。

\url{http://qiita.com/tuneyukkie/items/e7565fb0856e6a9f517d}


\chapter{ロボット関連}
\label{learned_robot:id1}\label{learned_robot::doc}\begin{quote}\begin{description}
\item[{著者}] \leavevmode
Masato

\end{description}\end{quote}


\section{概要}
\label{learned_robot:id2}
ロボットについて必要な知識や使い方までをまとめていきます。ハードウェア関連では製作に必要な力学系の知識から
ソフトウェア関連はロボットOSとして知られるROSのインストール、使い方までまとめていきます。


\section{力学系}
\label{learned_robot:id3}

\section{ROS}
\label{learned_robot:ros}

\chapter{電気関連}
\label{learned_electric:id1}\label{learned_electric::doc}\begin{quote}\begin{description}
\item[{著者}] \leavevmode
Masato

\end{description}\end{quote}


\section{概要}
\label{learned_electric:id2}
電気に必要な知識についてまとめていきます。主に電気主任技術社に向けたまとめになっていますが、電子回路の製作
についてもまとめていきます。


\chapter{ネットワーク関連}
\label{learned_network:id1}\label{learned_network::doc}\begin{quote}\begin{description}
\item[{著者}] \leavevmode
Masato

\end{description}\end{quote}


\section{概要}
\label{learned_network:id2}
Linuxに関わるものでないネットワーク関連についてまとめていきます。どちらかと言うと、理論よりの話になります。


\chapter{製作物関連}
\label{works:id1}\label{works::doc}\begin{quote}\begin{description}
\item[{著者}] \leavevmode
Masato

\end{description}\end{quote}


\section{概要}
\label{works:id2}
筆者が製作したものについてまとめていきます。主に電気回路系統やロボット関連、プログラミング関連になります。


\chapter{考え方}
\label{tips:id1}\label{tips::doc}\begin{quote}\begin{description}
\item[{著者}] \leavevmode
Masato

\end{description}\end{quote}


\section{概要}
\label{tips:id2}
日々感じたことや大事にしていきたい考え方について徒然にまとめていきます。


\section{引き継ぎ(ドキュメントの重要性)}
\label{tips:id3}

\subsection{引き継ぎものがないとは}
\label{tips:id4}
引き継ぎをちゃんとやらない人はビジネスパーソンとしての自覚が足りないと感じるし、仕事をしているにしても評価することができない。 \textbf{「引き継ぎがいらない」} が意味することは、本人が駄目であることを認めている。
仮に、引き継ぐことがないとすると、
\begin{itemize}
\item {} 
その本人が自分で工夫したこと、暗黙的な知的コツが全くない。

\item {} 
何も考えずに仕事をしていた。

\item {} 
全社最適視点、全プロジェクト視点がなく、自分仕事が片付けばいいと思っている。

\item {} 
他社への思いやりや仕事への愛情がない。

\end{itemize}

ということになる。また、「コツを完璧に文章化してある。」というケースもあるが、「どこに何が書いてあるのか」を丁寧に伝えないと、何が書いてあるのか分からないものになってしまう。「引き継ぎ資料の使い方を引き継ぐ」という感じに。


\subsection{引き継ぎ力=仕事力}
\label{tips:id5}
引き継ぎに必要な要素を考えてみると、かなりの超人である。

\textbf{1. 言語化能力}

引き継ぎで一番価値があるのは、「私は経験を積んだからできるが、引き継ぎ相手はできない」ことを、短期間で出来るようにするためにのアドバイスや情報である。つまり、自分が習得したコツを相手が分かるように言語化して伝える必要がある。

\textbf{2. 想像力、憑依力}

コツを相手が分かる形で伝えるのはなかなか難しい。
「初心者ならこういう状況に陥るだろうな」、
「この引き継ぎ相手なら、ここで躓くだろうな」
を想像したり、相手になってシミュレーションする力がないと、いい引き継ぎは出来ない。悪気はないのに、「特段、引き継ぐことはありません」という人はこの能力が決定的に足りない。

\textbf{3. 段取り、時間管理、リスク管理}

大抵の引き継ぎは、元々いた人が異動するなどで期限が決まっている。優れた引き継ぎは、期限に向けて着実に終わらせなければならない。突発の仕事が入ったので引き継ぎ作業は諦めるという訳にはいかない。
もちろん、自分の都合だけでなく、相手の理解スピードや割ける時間も計算にいれなければならない。

\textbf{4. 責任感、漢気、チームワーカーとしての心得}

想像力や段取り力が高い人でも、他人がいい仕事をする異に全く興味がなければ、きちんと引き継ぎをしない。
\begin{itemize}
\item {} 
折角だから、自分が苦労したところを説明していこう

\item {} 
去るのは心苦しいが、せめてノウハウを残そう

\item {} 
理屈はともかく、他人がいい仕事をするのがうれしい。

\end{itemize}

といった、他者を思いやることができる心意気を持って、仕事をすることは大切である。損得じゃないのだけれど、長期的には損得にも響く。

\textbf{5. 構想力}

本来引き継ぎは引き継ぐ人の工夫や根性でやるものではなく、組織として引き継ぎを計画していかなければならない。 \textbf{「自然に引き継がれる」} が理想。組織が意識することは重要であるが、個人の能力として必要なのは、
「今後、このしごとは、誰が、どうやってやっていくのか?」を考える力、つまり構想力が必要である。

将来のために引き継ぐのだから、将来を見通す力が必要である。
\textless{}\url{http://blogs.itmedia.co.jp/magic/2012/10/post-a14e.html}\textgreater{}


\section{Wikiに情報を書くときに守るべきルール}
\label{tips:wiki}

\subsection{ルール}
\label{tips:id6}
守るべきルールは、
\begin{quote}

「このページについて」という欄をページの先頭に用意し、そのページの概要を1~2文で書く。
\end{quote}

である。単純だが、強力で、きちんと守ろうとすると意外に面倒なルールである。

例えば、何かのインストール作業をメモしておきたい場合、「このページについて」欄には以下のような文章を書く。:

\begin{Verbatim}[commandchars=\\\{\}]
\PYGZsh{}\PYGZsh{} このページについて
これは、システムAの開発環境を構築するために、Windows7にソフトウェアBをインストールした際のメモです。

\PYGZsh{}\PYGZsh{} 手順
    ...
\end{Verbatim}

また、何かの新機能を実装したくて、事前に設計を検討した場合は、こんな感じで書きます。:

\begin{Verbatim}[commandchars=\\\{\}]
\PYGZsh{}\PYGZsh{} このページについて

これは、機能Cの設計についてチーム内で議論するために、個人的に作成した設計案です。設計の事前知識として、既存のソースコードを調査した結果も含めています。

\PYGZsh{}\PYGZsh{} 設計案
    ...
\end{Verbatim}


\subsection{ルールの目的}
\label{tips:id7}
このルールを徹底する目的は、そのページを開いた人に \textbf{「このページに、自分が今知りたい情報がかかれているか?」を瞬時に判断するため} の情報をあたえることである。
wikiなどを普段使っている人なら、このページはこんなことが書いてあるんだろうなと思って読み始めたら、実は全然違うことが書いてあった、という経験があると思う。例えば、
\begin{itemize}
\item {} 
インストール手順書かと思って読んでいたら、インストール時に試したことを適当な順序で書いてあるだけのメモであった。書いてある手順を順に試していくと、途中で失敗してしまった。

\item {} 
ある機能の設計に関するメモかと思ったら、単なる設計案のメモであり、最終的に実装されたソースコードと全く一致していなかった。

\end{itemize}

Wikiに色々まとめておくことは悪いことでないが、ただ、そのページがどういう状態なのかが明示されていないと、読み解いて、必要な情報ではなかったと判断する時間が無駄になる。
その結果、Wikiに書いてある情報自体を信用しなくなり、例え他に有益な情報が書いてあってもその情報は読まれないという状況を生む。

ページの先頭にただのメモであると明示してあれば、読み手が、 \textbf{「他に有益な情報が見当たらないから、これを頑張って読み取ろう」} と判断することも十分にありえ、
結果として、必要な情報がなくても、期待を裏切ることにはならない。


\subsection{ルールを守るために}
\label{tips:id8}\begin{itemize}
\item {} 
ページの更新が終わった段階で、「このページについて」の内容を見直す。

\item {} 
「このページについて」から外れる内容が増えてきたら、別のページに分ける。

\item {} 
他人が書いた「このページについて」も直していい。

\end{itemize}

\textless{}\url{http://iyoshiz.hatenablog.com/entry/2016/04/25/015644}\textgreater{}


\chapter{本}
\label{books:id1}\label{books::doc}\begin{quote}\begin{description}
\item[{著者}] \leavevmode
Masato

\end{description}\end{quote}


\section{概要}
\label{books:id2}
読んでいる本や読んだ本、これから読む本についてまとめていきます。


\section{読んでいるもの}
\label{books:id3}

\section{これから読むもの}
\label{books:id4}

\section{読んだもの}
\label{books:id5}

\chapter{イベント・勉強会関連}
\label{events:id1}\label{events::doc}\begin{quote}\begin{description}
\item[{著者}] \leavevmode
Masato

\end{description}\end{quote}


\section{概要}
\label{events:id2}
参加したイベントや勉強会で学んだことをまとめていきます。あとで調べるメモ用。


\chapter{美術関連}
\label{hobby_arts:id1}\label{hobby_arts::doc}\begin{quote}\begin{description}
\item[{著者}] \leavevmode
Masato

\end{description}\end{quote}


\section{概要}
\label{hobby_arts:id2}
美術展でおっ!っと思ったものや文化についてまとめていきます。


\chapter{言語}
\label{hobby_lang:id1}\label{hobby_lang::doc}\begin{quote}\begin{description}
\item[{著者}] \leavevmode
Masato

\end{description}\end{quote}


\section{概要}
\label{hobby_lang:id2}
英語やスペイン語での知らない単語帳やおもしろいと思った話をまとめていきます。


\chapter{SSまとめ}
\label{matome_ss:ss}\label{matome_ss::doc}\begin{quote}\begin{description}
\item[{著者}] \leavevmode
Masato

\end{description}\end{quote}


\section{概要}
\label{matome_ss:id1}
読んでいいと思ったSSについてまとめていきます。リンクをたどってください。


\chapter{Indices and tables}
\label{index:indices-and-tables}\begin{itemize}
\item {} 
\DUrole{xref,std,std-ref}{genindex}

\item {} 
\DUrole{xref,std,std-ref}{modindex}

\item {} 
\DUrole{xref,std,std-ref}{search}

\end{itemize}



\renewcommand{\indexname}{索引}
\printindex
\end{document}
